\documentclass[10pt]{article}
\usepackage[portuguese]{babel}
\usepackage[utf8]{inputenc}
\usepackage{graphicx}
\usepackage{hyperref}
\graphicspath{ {./Figuras/} }

\title{Descritivo do trabalho de Segurança Computacional 2021.1}
\author{Mateus Luis Oliveira - 180112490}
\date{Agosto 2021}

\begin{document}

\maketitle

\section{Introdução}
A cifra de Vigenère é um exemplo de cifra polialfabética, que é uma cifra baseada 
na substituição, usando vários alfabetos de substituição. Nestre trabalho iremos tratar da implementação de um cifrador/decifrador/quebrador 
de cifra de Vigenère.



\section{Implementação}

\subsection{Arquitetura do Projeto}
Para definir a forma que o projeto foi estruturado decidi tomar como base uma interface de usuário criada pelo RAD tool (Rapid-application development) 
\href{https://github.com/wxFormBuilder/wxFormBuilder}{wxFormBuilder}, que gera uma inteface de usuário usando como base os objetos de inteface
\href{https://www.wxpython.org/}{wxPython}. Um arquivo de interface é gerado dentro da ferramenta contendo todas as classes de telas (frames) e 
outros objetos que compoem a interação com usuário, como botões e caixas de texto, conforme descrito na figura \ref{fig:wxform}.

\begin{figure}[h]
  \includegraphics[scale=0.1]{wxform}
  \centering
  \caption{Projeto criado no wxFormBuilder.}
  \label{fig:wxform}
\end{figure}
O arquivo gerado pelo wxFormBuilder usado como base no projeto é o aqruivo ../scgui.py, que é importado dentro do arquivo principal do projeto,
../main.py, para que seus componentem sejam consumidos.

\subsection{Desenvolvimento}


Dentro da implementação do projeto foi usado como base uma estrutura de dados
criada de acordo com o código descrito na figura \ref{fig:matriz_code}.

\begin{figure}[h]
    \includegraphics[scale=0.5]{Matrix_code}
    \centering
    \caption{Implementação da matriz de Vigenère.}
    \label{fig:matriz_code}
  \end{figure}

A estrutura consiste em uma lista de listas que contém todas as 26 letras do alfabeto. Onde cada lista varia apenas a letra de início.
Usando essa matriz a cifração e decifração se torna mais simples, pois conseguimos acessar a letra cifrada ou decifrada com índices.\\
Também foi criado uma estrutura mais básica para consulta do índice de acordo com a posição da letra do alfabeto, como demonstrado na figura 
\ref{fig:alfabeto_code} .

\begin{figure}[h]
  \includegraphics[scale=0.5]{dicAlfabeto}
  \centering
  \caption{Implementação de um dicionário contendo as letras do alfabeto.}
  \label{fig:alfabeto_code}
\end{figure}

\newpage
\subsubsection{Cifração}
A cifração foi implementada de acordo com o descrito no código da figura \ref{fig:cifrador}.
Onde podemos observar o uso da indexação da matriz criada na figura \ref{fig:matriz_code} em vermelho, usando como base o dicionário
do alfabeto criado na figura \ref{fig:alfabeto_code} em azul.


\begin{figure}[h]
  \includegraphics[scale=0.4]{cifrador}
  \centering
  \caption{Implementação do cifrador Vigenère.}
  \label{fig:cifrador}
\end{figure}
Também podemos observar nos pontos marcados em rosa na figura \ref{fig:cifrador} o uso de um keystream que é criado de acordo com a função
createKeyStream, que usa como base o texto original não cifrado de tamanho N para criar um texto de repetição de key também de tamanho N.
\subsubsection{Decifração}
Para a decifração foi implementado um algoritmo que gera também um keystream baseado na key que o usuário passa e usa as mesmas estruturas
descritas na figura \ref{fig:cifrador} para fazer o processo de tradução/cifração. O código pode ser conferido na figura \ref{fig:decifrador}.

\begin{figure}[h]
  \includegraphics[scale=0.4]{decifrador}
  \centering
  \caption{Implementação do decifrador Vigenère.}
  \label{fig:decifrador}
\end{figure}

\subsubsection{Quebra da cifra por análise de frequências}

Para conseguirmos decifrar um texto não sabendo sua chave temos que dividir a quebra em algumas etapas.
O primeiro passo é inferir o tamanho da chave através do seguinte método:
Listamos todas as sequências de 3 caracteres que se repetem pelo menos 2 vezes no texto e traçamos todas as distâncias desses 
seqências entre elas


Dado uma chave K que é usada para cifrar um texto, tendo conhecimento do tamanho da chave, podemos inferir que 
para cada caracter de K as frequências relativas do texto se mantêm, por conta das propiedades do keystream,
fazendo uma troca do índice do alfabeto. Como explicitado no exemplo da figura \ref{fig:ex}:

\begin{figure}[h]
  \includegraphics[scale=0.2]{exemplo}
  \centering
  \caption{Exemplo de uma implementação do keystream usando uma chave KEY.}
  \label{fig:ex}
\end{figure}

Usando como base esta key de tamanho 3, podemos calcular a distribuição de freqência iterando a enésima letra do texto cifrado de acordo com
keystream. Então calcularemos a distribuição de todas as frequências que tem como base o caracter 'k' da keystream, depois calculamos para 'E', 
e em seguida calculamos para 'y'.\\
Por último analisamos as frequências e trocando os índices conseguimos inferir a letra correta do keystream.
Todo o processo de quebra está explicado detalhadamento neste vídeo que pode ser acessado através deste \href{https://youtu.be/NIzf1nHWtfs}{link}.

\end{document}
