\documentclass{article}
\usepackage[portuguese]{babel}
\usepackage[utf8]{inputenc}
\usepackage{graphicx}
\usepackage{hyperref}
\graphicspath{ {./Figuras/} }

\title{Descritivo do trabalho de Segurança Computacional 2021.1}
\author{Mateus Luis Oliveira}
\date{Agosto 2021}

\begin{document}

\maketitle

\section{Introdução}
A cifra de Vigenère é um exemplo de cifra polialfabética, que é uma cifra baseada 
na substituição, usando vários alfabetos de substituição. Nestre trabalho iremos tratar da implementação de um cifrador/decifrador/quebrador 
de cifra de Vigenère.



\section{Implementação}

\subsection{Arquitetura do Projeto}
Para definir a forma que o projeto foi estruturado decidi tomar como base uma interface de usuário criada pelo RAD tool (Rapid-application development) 
\href{https://github.com/wxFormBuilder/wxFormBuilder}{wxFormBuilder}, que gera uma inteface de usuário usando como base os objetos de inteface
\href{https://www.wxpython.org/}{wxPython}. Um arquivo de interface é gerado dentro da ferramenta contendo todas as classes de telas (frames) e 
outros objetos que compoem a interação com usuário, como botões e caixas de texto, conforme descrito na figura \ref{fig:wxform}.

\begin{figure}[h]
  \includegraphics[scale=0.1]{wxform}
  \centering
  \caption{Projeto criado no wxFormBuilder.}
  \label{fig:wxform}
\end{figure}
O arquivo gerado pelo wxFormBuilder usado como base no projeto é o aqruivo ../scgui.py, que é importado dentro do arquivo principal do projeto
para que seus componentem sejam consumidos pelo arquivo principal ../main.py.

\subsection{Desenvolvimento}


Dentro da implementação do projeto foi usado como base uma estrutura de dado
Criada de acordo com o código descrito na figura \ref{fig:matriz_code}.

\begin{figure}[h]
    \includegraphics[scale=0.5]{Matrix_code}
    \centering
    \caption{Implementação da matriz de Vigenère.}
    \label{fig:matriz_code}
  \end{figure}

A estrutura consiste em uma lista de listas que contém todas as 26 letras do alfabeto. Onde cada lista varia apenas a letra de início.
Usando essa matriz a cifração e decifração se torna mais simples, pois conseguimos acessar a letra cifrada ou decifrada com índices.

\subsubsection{Cifração}
\subsubsection{teste2}

\end{document}
